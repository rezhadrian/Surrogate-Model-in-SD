\newpage
\section{Random Dynamic System}

The steady-state response of a Linear Time-Invariant (LTI) structural dynamic system can be found through solving the discretized equation of motion in the frequency domain instead of time domain.
The equation of motion in time and frequency domain are given below. 

\begin{equation}
    \mathbf{M}\ddot{\mathbf{u}}(t) +
    \mathbf{C}\dot{ \mathbf{u}}(t) +
    \mathbf{K}      \mathbf{u} (t)  =
    \mathbf{f}(t)
    \phantom{x}
    \xrightarrow{
        \phantom{x}
        \mathscr{F}
        \phantom{x}
    }
    \phantom{x}
    \left(
        \mathbf{K} - \omega^{2}\mathbf{M} +
        j\omega \mathbf{C}
    \right)
    \mathbf{U}(\omega) =
    \mathbf{F}(\omega)
\label{EoM}
\end{equation}

Equation \ref{EoM} depends on structural and load parameters, which are subject to uncertainties. 
To account for this, the problem can be stated in a probabilistic setting, and the system of equations is reformulated as a function of a set of random variables $\mathbf{x}$: 

\begin{equation}
    \left[
        \mathbf{K}(\mathbf{x}) - 
        \omega^{2}\mathbf{M}(\mathbf{x}) +
        j\omega \mathbf{C}(\mathbf{x})
    \right]
    \mathbf{U}(\omega,\mathbf{x}) =
    \mathbf{F}(\omega,\mathbf{x})
\label{Random EoM}
\end{equation}

One can obtain distribution of $\mathbf{U}(\omega,\mathbf{x})$ using Monte Carlo Simulation (MCS) i.e. to solve the equations multiple times, each time using a different sample of $\mathbf{x}$. 
For large and complex system, the computational cost of MCS is too high. 
Thus, one usually create a meta / surrogate model that approximate the response and is cheaper to compute. 

